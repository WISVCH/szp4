% Team informatie voor programmeerwedstrijden

\documentclass[a4paper]{artikel3}

\usepackage[dutch]{babel}
\usepackage{pslatex,geometry}
\usepackage[latin1]{inputenc}
\usepackage[T1]{fontenc}
\geometry{noheadfoot,left=0.75in,right=0.75in,top=1in,bottom=1in}
\pagestyle{empty}

\catcode`\|\active
\def|#1|{\textsf{\textbf{#1}}}
\renewcommand{\rmdefault}{bch}

\hyphenpenalty=10000

\begin{document}

\section*{Richtlijnen voor het inzenden van programma's}

Ingestuurde programma's worden op de computers van de jury automatisch
gecompileerd en nagekeken. Dit gebeurt altijd onder
GNU/Linux. Hieronder geven we aan op welke manier de programma's
geacht worden hun invoer te lezen en uitvoer te schrijven. Verder
geven we per programmeertaal informatie over de compiler en over de
toegestane taalconstructies. Voor een goed verloop van het
nakijk-proces, is het belangrijk dat jullie programma's zich exact aan
de hier beschreven richtlijnen houden. Een kleine vergissing,
bijvoorbeeld in de naam van het invoerbestand, en de inzending wordt
onherroepelijk fout gerekend!

Om programma's te compileren met dezelfde instellingen als die van de jury,
hebben we een Makefile geschreven. Je hoeft hierbij alleen maar |make|
te typen om programma's te compileren. De sourcebestanden moeten dan
wel de juiste extensie hebben: |.c| voor C, |.cc| voor C++ en |.java|
voor Java.

Documentatie voor C, C++ en Java is beschikbaar in HTML, te lezen met
Firefox. Bovendien zijn de |man|pages voor C ge\"installeerd.

Let op, jullie programma's mogen geen gebruik maken van:
\begin{itemize}
\item assembler code, hetzij inline of op een andere manier,
\item netwerktoegang,
\item bestandstoegang anders dan de aangegeven invoerfile,
\item procesmanipulatie (|system()|, |fork()|,
  |exec()|, |kill()| en dergelijke),
\item meer dan 512MB geheugen.
\end{itemize}

Het is jullie eigen verantwoordelijkheid om ervoor te zorgen dat de
ingestuurde programma's voldoen aan de bovengenoemde regels. Bij het
opzettelijk overtreden van de regels kan de jury het betreffende team
diskwalificeren.

Bij het gebruik van meer dan 512 MB geheugen of het geven van meer dan 64MB
output, krijg je een \emph{runtime error}.

\subsection*{Invoer en uitvoer}

Invoer wordt gelezen vanuit een bestand met als naam de letter van de
opgave, gevolgd door |'.in'|; alles in kleine letters. Zo staat de
invoer voor opgave C dus in het bestand |'c.in'|. Uitvoer wordt
geschreven naar de standaard-uitvoer (naar het 'scherm'). Let erop dat
je exact het formaat aanhoudt dat in de opgave wordt
beschreven. Gebruik \emph{geen hoofdletters} in de naam van het
invoerbestand, bestandsnamen op GNU/Linux zijn namelijk
hoofdlettergevoelig.

Het lezen of schrijven van andere bestanden dan deze is niet toegestaan. Het
lezen van standaard-invoer zal resulteren in een \emph{run time limit
exceeded} omdat jullie programma zal blijven wachten op invoer van het
toetsenbord (wat de jury niet zal gaan invoeren).

\subsection*{C / C++}

\begin{tabular}{rl}
Compiler: & GNU GCC 4.3.1 \\
C opties: & {\tt gcc -Wall -O2 -g -std=c99 -lm} \\
C++ opties: & {\tt g++ -Wall -O2 -g -std=c++98 -lm} \\
\end{tabular}

Zorg ervoor dat jullie programma een \emph{exit code} van 0 geeft. In
Java gebeurt dit automatisch; in C of C++, declareer |main| als |int|
en gebruik een expliciet |return 0;| of |exit(0);| statement. Als
|main| als |void| gedeclareerd wordt, kan een waarde anders dan 0
teruggegeven worden, wat een \emph{runtime error} melding kan
opleveren.

Gebruik \emph{geen binary mode} voor het lezen van invoerbestanden.
Schrijf dus niet |fopen("{}invoer", "rb")|, maar gewoon
|fopen("{}invoer", "r")|.

In C mag alleen gebruikt gemaakt worden van functies die gedefinieerd worden
door de ISO C99 standaard. Hieronder vallen de volgende headers:
|<assert.h>|, |<complex.h>|, |<ctype.h>|, |<errno.h>|, |<fenv.h>|,
|<float.h>|, |<inttypes.h>|, |<iso646.h>|, |<limits.h>|,
|<locale.h>|, |<math.h>|, |<setjmp.h>|, |<signal.h>|, |<stdarg.h>|,
|<stdbool.h>|, |<stddef.h>|, |<stdint.h>|, |<stdio.h>|, |<stdlib.h>|,
|<string.h>|, |<tgmath.h>|, |<time.h>|, |<wchar.h>|, |<wctype.h>|.

Voor C++ gelden dezelfde regels als voor C, maar in dit geval mogen ook
de standaard C++ headers gebruikt worden. Dit zijn: |<algorithm>|,
|<bitset>|, |<deque>|, |<exception>|, |<fstream>|, |<functional>|,
|<iomanip>|, |<ios>|, |<iosfwd>|, |<iostream>|, |<istream>|, |<iterator>|,
|<list>|, |<map>|, |<memory>|, |<new>|, |<numeric>|, |<ostream>|, |<queue>|,
|<set>|, |<sstream>|, |<stack>|, |<stdexcept>|, |<streambuf>|, |<typeinfo>|,
|<utility>|, |<valarray>| en |<vector>|. Alle C-headers zijn ook
toegestaan.

Andere header files mogen niet gebruikt worden. |\#pragma| mag ook niet
gebruikt worden. Het gebruik van STL is \emph{wel} toegestaan.

Het gebruik van de C++ stream operators \verb|<<| en \verb|>>| zorgt ervoor
dat je programma trager wordt, dan wanneer je de C-routines zou gebruiken.

\subsection*{Java}

\begin{tabular}{rl}
Compiler: & JDK 1.6.0 \\
Opties: & {\tt javac -O} \\
\end{tabular}

In Java zijn de volgende packages toegestaan: |java.io|, |java.lang|,
|java.text|, |java.math| en |java.util|. Andere packages mogen dus niet.

Het is toegestaan om meerdere \emph{classes} te gebruiken in je programma,
maar deze moeten wel allemaal in een enkel |.java| bestand gedefinieerd
worden. In dat geval mag je de classes niet als |public| definieren. E\'en
van je classes moet een |main| method bevatten. De naam van deze class
\emph{moet altijd} exact gelijk zijn aan |Problem| gevolgd door een
hoofdletter die hoort bij het probleem wat je hebt opgelost.

Voor het lezen van invoerbestanden is het handig om gebruik te maken van de
class |StreamTokenizer|. Schrijven naar de uitvoer kan met de methode
|System.out.print|. Bijvoorbeeld:

\begin{verbatim}
import java.io.*;
class ProblemA {
    public static void main(String[] args) {
        try {
            // Open invoerbestand
            StreamTokenizer f = new StreamTokenizer(new FileReader("a.in"));
            f.resetSyntax();
            f.whitespaceChars(0, 0x20);  f.wordChars(0x21, 0xff);
            f.nextToken();  String s = f.sval;
            f.nextToken();  int i = Integer.parseInt(f.sval);
            System.out.println("Hallo daar.");
        } catch (IOException e) { System.err.println(e); }
    }
}
\end{verbatim}

\section*{Het GNU/Linux systeem tijdens de wedstrijd}

\subsection*{Inloggen}

Na het opstarten van de computer verschijnt het |gdm| login scherm,
welke binnen 30 seconden automatisch de gebruiker |team| inlogt. De
standaardomgeving die gestart wordt is GNOME. Wil je een andere window
manager, log dan uit en kies bij gdm je window manager naar keuze. Log
vervolgens in met gebruikersnaam |team| en wachtwoord |team|.

Het is niet mogelijk om op het console (in tekst-mode) te werken omdat
TeamTool een grafische browser nodig heeft. Na het inloggen kan je het
beste meteen Firefox starten. Deze opent automatisch de TeamTool
pagina, waar je kan inloggen met de username en wachtwoord die gegeven
is.

\subsection*{Printen}

Printen is mogelijk met het |a2ps| programma vanaf de command-line: \\ {\tt
a2ps \it filenaam}

De meeste editors kunnen geopende bestanden ook direct vanuit de editor
printen. Geprinte documenten worden door een runner naar je werkplek
gebracht; loop \emph{niet} zelf naar de printer om het document op te halen.

\subsection*{Documentatie}

Via het |man| commando kun je documentatie van het systeem en de C
library opvragen. Documentatie voor Java is beschikbaar in HTML, te
lezen met Firefox. Er zijn bookmarks naar de beschikbare documentatie.

\end{document}
