% Team informatie voor programmeerwedstrijden
% English version

\documentclass[a4paper]{artikel3}

\usepackage[dutch]{babel}
\usepackage{pslatex,geometry}
\usepackage[latin1]{inputenc}
\usepackage[T1]{fontenc}
\geometry{noheadfoot,left=0.75in,right=0.75in,top=1in,bottom=1in}
\pagestyle{empty}

\catcode`\|\active
\def|#1|{\textsf{\textbf{#1}}}
\renewcommand{\rmdefault}{bch}

\hyphenpenalty=10000

\begin{document}

{\bf \Large Guidelines for submitting solutions}
\bigskip

Submitted programs will be automatically compiled and checked on the judges'
computers. This will always be done on a Linux machine, even when your team
uses a Windows computer. This document will describe to you how your program
should read its input and how it should write its output. We will also
tell you about the compiler and language constructs you're allowed to use.
Please follow the guidelines carefully, since small mistakes will mean that
your solution will be rejected.

We've created a few scripts which you can use to compile your program. These
scripts use the same compiler flags the judges use. The scripts are called
|makec|, |makecpp|, |makepas| and |makejava|. They take exactly one
argument: the name of your program, without the extension.

There's documentation available for C, FreePascal and Java. Firefox has
bookmarks to the appropriate locations. On the Linux machines, we've also
installed the |man|pages for C.

Your program is not allowed to use:
\begin{itemize}
\item assembly code,
\item network access,
\item file access to files other than the input file,
\item process manipulation (|system()|, |fork()|,
  |exec()|, |kill()| and the likes),
\item more than 256MiB of memory.
\end{itemize}

It's your own responsibility to make sure your programs comply with the
above rules. The judges can disqualify your team if it disregards the rules.

If your program uses more than 256 MiB of memory, or writes more than 64 MiB
output, the result will be a \emph{runtime error}.

\section*{Input and output}

The name of the input file is the letter of the problem, followed by
|'.in'|. The whole filename should be in lowercase! For example, input for
problem C is supplied in the file called |'c.in'|. Make sure you use the
format just described. \emph{Do not use capitals} in the filename. This will
work under Windows, but it will fail on the judges' machines.

Reading from or writing to other files is not allowed. If you read from
stdin (the ``keyboard''), you will get a \emph{run time exceeded}, because
your program will be waiting for input, which it won't get.

Linux and Windows use different line endings. If you're working under
Windows, be mindful of this. Your program will be judged on a Linux machine.

\section*{C / C++}

\begin{tabular}{rl}
Compiler: & GNU GCC 3.3.5 \\
C flags: & {\tt gcc -Wall -O2 -g -std=c99 -lm} \\
C++ flags: & {\tt g++ -Wall -O2 -g -std=c++98 -lm} \\
\end{tabular}

Make sure your programs gives \emph{exit code} 0. You should declare your
|main| function as an |int| and explicitly use |return 0;| or |exit(0);|. If
you declare |main| as |void|, the \emph{exit code} can be something
different from 0, which will result in a \emph{runtime error}.

Don't use binary mode for reading of input files, e.g. don't use
|fopen("{}a.in", "rb")|, but |fopen("{}a.in", "r")|.

In C you can only use the following headers:
|<assert.h>|, |<complex.h>|, |<ctype.h>|, |<errno.h>|, |<fenv.h>|,
|<float.h>|, |<inttypes.h>|, |<iso646.h>|, |<limits.h>|,
|<locale.h>|, |<math.h>|, |<setjmp.h>|, |<signal.h>|, |<stdarg.h>|,
|<stdbool.h>|, |<stddef.h>|, |<stdint.h>|, |<stdio.h>|, |<stdlib.h>|,
|<string.h>|, |<tgmath.h>|, |<time.h>|, |<wchar.h>|, |<wctype.h>|.

The same rules apply for C++ as apply to C, but you can also use the
following C++ headers: 
|<algorithm>|, |<bitset>|, |<deque>|, |<exception>|, |<fstream>|,
|<functional>|, |<iomanip>|, |<ios>|, |<iosfwd>|, |<iostream>|, |<istream>|,
|<iterator>|, |<list>|, |<map>|, |<memory>|, |<new>|, |<numeric>|,
|<ostream>|, |<queue>|, |<set>|, |<sstream>|, |<stack>|, |<stdexcept>|,
|<streambuf>|, |<typeinfo>|, |<utility>|, |<valarray>| and |<vector>|.

The usage of other header files is not allowed. You are not allowed to use
|\#pragma|. Using the STL is allowed.

Please be careful when using the C++ stream operators \verb|<<| and
\verb|>>|, these will make your program run slower than if you would use the
equivalent C-functions.

\section*{Pascal}

\begin{tabular}{rl}
Compiler: & FreePascal 2.0.0 \\
Flags: & {\tt fpc -Sgh2 -O1 -Cr-t- -g} \\
\end{tabular}

In Pascal you can use the units |classes|, |getopts|, |math|, |objects|,
|objpas|, |strings|, |strutils|, |system| and |sysyutils|. You can also use
the boolean directives |\$H|, |\$I|, |\$P|, |\$Q|, |\$R|, |\$T|, |\$X| and
|\$STATIC|, and the directives |\$DEFINE|, |\$ELSE|, |\$ENDIF|, |\$IF|,
|\$IFDEF|, |\$IFNDEF|, |\$IFDEF|, |\$MACRO|, |\$MODE| and |\$UNDEF|. You are
not allowed to use other units and directives.

Please use the standard |ReadLn|, |Read| and |Eoln| functions for reading
input files. Don't assume the lines are closed by a certain character, like
|\#10| or |\#13\#10|.

The {\tt -S2} compiler flag sets the |OBJFPC| mode. If you want to use a
different mode, like |TP| or |DELPHI|, you can set it using the |\$MODE|
directive.

The FreePascal IDE is available. The command to run it is |fp-ide| (when
you're using Linux; it will open in a new xterm) or |fp| (when you're using
Windows). The |Alt|-key may not work. You can use |Esc| as a prefix to
replace it, e.g. |Esc| followed by |X| will do the same as |Alt|+|X|.

\section*{Java}

\begin{tabular}{rl}
Compiler: & JDK 1.5.0 \\
Flags: & {\tt javac -O} \\
\end{tabular}

In Java you can use the packages |java.io|, |java.lang|, |java.text|,
|java.math| and |java.util|. Using other packages is not allowed.

You can use multiple classes in your program, but you'll need to define them
all in a single |.java| file. If you do this, you can't define these classes
as being |public|.

One of your classes should have a |main| method. The name of this class
should always be |Problem| followed by a capital letter indicating the
problem you solved. For example: your class that solves problem A should be
called |ProblemA|. Calling it differently will result in either a
\emph{compiler error} (if your class is |public|) or a \emph{runtime error}
(if your class is package private, i.e. when you haven't given an access
specifier).

For reading input files you can use the class |StreamTokenizer|. You can
write output with |System.out.print| or |System.out.println|.

\begin{verbatim}
import java.io.*;

public class ProblemA {
    public static void main(String[] args) {
        try {
            // Open input file
            StreamTokenizer f = new StreamTokenizer(new FileReader("a.in"));
            f.resetSyntax();
            f.whitespaceChars(0, 0x20);
            f.wordChars(0x21, 0xff);
            
            // Read a word
            f.nextToken();
            String s = f.sval;
            
            // Read an integer
            f.nextToken();
            int i = Integer.parseInt(f.sval);
            
            // Write a line of output
            System.out.println("Hello world!");
        } catch (IOException e) {
            System.err.println(e);
        }
    }
}
\end{verbatim}

\vspace*{1cm}
{\bf \Large The Linux system during the contest}

\section*{Logging in}

After booting the computer a |kdm| login screen appears. You can log in
using the username |szp| and the password |team|. If you want you can choose
a different window manager. Do this before logging in.

Don't log in on the console (text mode), you'll need a graphical browser to
use the contest system.

It's recommended to start Firefox directly after logging in. Firefox will
automatically open TeamTool.

\section*{Printing}

You can print using |lpr| from the command line: \\
{\tt lpr \it filename}

Most editors are capable of printing opened files directly. Printed
documents will be brought to your computer by a runner. You are not allowed
to go to the printer to retrieve the document.

\section*{Documentation}

Documentation for the C library and the system is available using the |man|
command. FreePascal and Java documentation is available in HTML. Firefox
contains bookmarks to the relevant locations.

\vspace*{1cm}
{\bf \Large The Windows system during the contest}

\section*{Logging in}

After booting you'll see a Windows XP login screen. Login to the account
|szp| using |team| as password.

It's recommended to start Firefox directly after logging in. Firefox will
automatically open TeamTool.

\section*{Saving files}

You can save your files in |c:\string\team|. You can create sub-directories
if you wish. You can also save them to My Documents. Don't use other
locations for your files.

\section*{Printing}

You can use the default Windows printer driver (Generic Postscript Printer)
to print your files. Printed documents will be brought to your computer by a
runner. You are not allowed to go to the printer to retrieve the document.

\section*{Documentation}

Documentation for C, FreePascal and Java is available in HTML. Firefox has
bookmarks to the relevant locations.

\end{document}
