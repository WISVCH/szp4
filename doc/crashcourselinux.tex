\documentclass[a4paper]{artikel3}

\usepackage[dutch]{babel}
\usepackage{pslatex,geometry}
\usepackage[latin1]{inputenc}
\usepackage[T1]{fontenc}
\geometry{noheadfoot,left=0.75in,right=0.75in,top=1in,bottom=1in}
\pagestyle{empty}

\catcode`\|\active
\def|#1|{\textsf{\textbf{#1}}}
\renewcommand{\rmdefault}{bch}

\hyphenpenalty=10000

\begin{document}

\section*{Crash course programmeren voor DKP onder GNU/Linux}

Programma's schrijven en compileren bestaat doorgaans uit die twee simpele
stappen (IDEs als Eclipse daargelaten). Voor het eerste gedeelte, schrijven,
zijn er verschillende editors beschikbaar. Vim en Emacs zijn aanwezig voor de
liefhebbers, de rest raden we Kate of Gedit aan. 

De volledige lijst beschikbare editors is te vinden onder
Applications$\rightarrow$Debian$\rightarrow$Apps$\rightarrow$Editors
.

Als je je programma eenmaal geschreven hebt sla deze dan op met de juiste
extensie, dat wil zeggen |.c| voor C, |.cc| voor C++ en |.java| voor Java. Om
het zaakje te compileren open je een terminal (de Terminal snelkoppeling op de
desktop of via het menu
Applications$\rightarrow$Accessories$\rightarrow$Terminal ). Type in je
terminal |make| in. Make zal alle source files in de directory compileren tot
programma's.

Om te controleren of het programma doet wat het moet doen kan je het uitvoeren
door |./programma| in te geven, of (natuurlijk) |java ProblemA| in het geval
van java.

Mocht je de uitvoer op willen vangen om rustig na te kijken, dirigeer het dan
naar file: |./programma > output| en open |output| in je editor.

\end{document}
